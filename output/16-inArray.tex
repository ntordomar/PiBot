\documentclass{article}
\usepackage[utf8]{inputenc}
\usepackage[spanish]{babel}
\usepackage{graphicx}
\usepackage{float}
\usepackage{amsmath}
\usepackage{amssymb}
\usepackage{amsfonts}
\usepackage{listings}
\usepackage{color}
\usepackage{hyperref}
\usepackage{tikz}
\usepackage{pgfplots}
\usepackage{pgfplotstable}
\usepackage{booktabs}
\usepackage{multirow}
\usepackage{array}
\usepackage{longtable}
\usepackage{tabularx}
\usepackage{pdflscape}
\usepackage{fancyhdr}
\date{}
\usepackage[margin=1in]{geometry}
\usepackage{varwidth}
\usepackage{setspace}
\newcommand{\multiline}[1]{\begin{varwidth}{\linewidth}$\displaystyle #1$\end{varwidth}}
\begin{document}
\begin{landscape}
\pagestyle{fancy}
\fancyhf{}
\renewcommand{\headrulewidth}{0pt}
\begin{flushleft}
\doublespacing
{\fontsize{14}{12}\selectfont
\textbf{\huge{Relational Algebra}}\newline \\
$\pi_{\text{NOMBRE}}$
\\
\hspace{0.5em}\fontsize{14}{12}\selectfont$\sigma_{\fontsize{10}{5}\multiline{\text{empleados.edad} = \text{''40''}\ OR\ \text{empleados.edad} = \text{''41''}\ OR\ \text{empleados.edad} = \text{''42''}}}$
\textbf{(empleados)}
}
\end{flushleft}
\vfill
Why did the relational algebra expression break up with the automaton? Because it couldn't handle its non-deterministic behavior and wanted a more SELECTive relationship!
\\ \\ \\ 
\begin{center}
\parbox{\linewidth}{\raggedright PiBot License v0.2.0\hfill Copyright © 2023 PiBot}
\end{center}
\end{landscape}
\end{document}
